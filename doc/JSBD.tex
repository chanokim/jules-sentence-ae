\documentclass[11pt,a4paper,halfparskip]{scrartcl}
\usepackage{graphicx}
\usepackage[utf8x]{inputenc}
\usepackage{url} 
\pagestyle{plain}

\title{\small{Documentation for}\\\huge JULIE Lab Sentence Boundary Detector\\\vspace{3mm}\small{Version 2.4}}


\author{\normalsize Katrin Tomanek\\
  \normalsize  Jena University Language \& Information Engineering (JULIE) Lab\\
  \normalsize F\"urstengraben 30 \\
  \normalsize D-07743 Jena, Germany\\
  {\normalsize \tt katrin.tomanek@uni-jena.de} }


\date{}

\begin{document}
\maketitle
\tableofcontents

\section{UIMA-Wrapper}



The JULIE Lab Sentence Boundary Detector (UIMA-JSBD) is a sentence
boundary detector for UIMA.  It is part of the JULIE Lab NLP tool
suite\footnote{\url{http://www.julielab.de/}} which contains several
UIMA-compliant NLP components from sentence splitting to named entity
recognition and normalization as well as a comprehensive UIMA type
system.

UIMA-JSBD is is an UIMA wrapper for JSBD, the respective command-line
version. For more detailed information on the functioning of JSBD
check the JSBD documentation or refer to \cite{Tomanek2007a}.

\subsection{Installation}

UIMA-JSBD comes as a UIMA pear file. Run the Pear-Installer (e.g.,
\url{./runPearInstaller.sh} for Linux) from your UIMA-bin directory.
After installation, you will find a subfolder \url{desc} in you
installation folder. This directory contains a descriptor
\url{SentenceAnnotator.xml} for UIMA-JSBD. You may now e.g. run UIMA's
Collection Proeccessing Engine Configurator (\url{cpeGUI.sh}) and add
UIMA-JSBD as a component into your NLP pipeline.

This pear package also contains a model for sentence splitting. The
model was trained on a special bio-medical corpus which consists of
data from both the GENIA \cite{Ohta2002} and the
\textsc{PennBioIE}\footnote{\url{http://bioie.ldc.upenn.edu/}} corpus
and additional material which we took from MedLine abstracts.
Currently, it comprises about 62000 sentences. An accuracy of 99.8\%
is yielded on this data using 10-fold cross-validation.  You will find
the model trained on this data in the directory \url{resources}.


\subsection{Requirements and Dependencies}

% mostly our tools will be based on java 1.5 and use UIMA
UIMA-JSBD is written in Java (version 1.5 or above required) using
Apache UIMA version
2.2.x-incubation\footnote{\url{http://incubator.apache.org/uima/}}.

% ref to our type system
The input and output of an AE takes place by annotation objects. The
classes corresponding to these objects are part of the \emph{JULIE Lab
  UIMA Type System} in its current version (2.1).\footnote{The
  \emph{JULIE Lab UIMA type system} can be separately obtained from
  \url{http://www.julielab.de/}, however, this package already
  includes the necessary parts of the type system.}

This version of UIMA-JSBD is based on JSBD-2.4 which employs the
machine learning toolkit MALLET \cite{McCallum2002}.




\subsection{Using the AE -- Descriptor Configuration}
% carefully edit this section!

In UIMA, each component is configured by a descriptor in XML. In the
following we describe how the descriptor required by this AE can be
created with the \emph{Component Descriptor Editor}, an Eclipse plugin
which is part of the UIMA SDK.

A descriptor contains information on different aspects. The following
subsection refers to each sub aspect of the descriptor which is, in
the Component Descriptor Editor, a separate \emph{tabbed page}. For an
indepth description of the respective configuration aspects or tabs,
please refer to the \emph{UIMA SKD User's
  Guide}\footnote{\url{http://incubator.apache.org/uima/}}, especially
the chapter on ``Component Descriptor Editor User's Guide''.

To define your own descriptor go through each tabbed pages mentioned
here, make your respective entries (especially in page \emph{Parameter
  Settings} you will be able to configure JNET to your needs) and save
the descriptor as \url{SomeName.xml}.

Otherwise, you can of course employ the descriptor that is contained
in the pear package you downloaded (in your installation directory, see
\url{desc/SentenceAnnotator.xml}).

\paragraph{Overview}
This tab provides general informtion about the component. For 
UIMA-JSBD you need to provide the information as specified in Table
\ref{tab:overview}.
% adapt to your needs, remember to change values in tabular below!

\begin{table}[h!]
  \centering
  \begin{tabular}{|p{3.5cm}|p{4cm}|p{6cm}|}
    \hline
    Subsection & Key & Value \\
    \hline\hline
    Implementation Details & Implementation Language & Java \\
    \cline{2-3}
    & Engine Type & primitive \\
    \hline
    Runtime Information & updates the CAS & check \\
    \cline{2-3}
    & multiple deployment allowed & check \\
    \cline{2-3}
    & outputs new CASes &  don't check \\
    \cline{2-3}
    & Name of the Java class file & \url{de.julielab.jules.ae.SentenceAnnotator}\\
    \hline
    Overall Identification Information & Name &  Sentence Annotator \\
    \cline{2-3}
    & Version &  2.4 \\
    \cline{2-3}
    & Vendor & JULIE Lab\\
    \cline{2-3}
    & Description & not needed\\
    \hline
  \end{tabular}
  \caption{Overview/General Settings for AE.}
  \label{tab:overview}
\end{table}


\paragraph{Aggregate}
% for primitive AEs this does not have to be set
Not needed here, as this AE is a primitive.

\paragraph{Parameters}
\label{sss:parameters}
% adapt to your needs

See Table \ref{tab:parameters} for a specification of the
configuration parameters of this AE. Do not check ``Use Parameter
Groups'' in this tab.

\begin{table}[h!]
  \centering
  \begin{tabular}{|p{4cm}|p{2cm}|p{2cm}|p{2cm}|p{4cm}|}
    \hline
    Parameter Name & Parameter Type & Mandatory & Multivalued & Description \\
    \hline\hline
    ModelFilename & String & yes & no & filename of trained model for
    JSBD\\
    \hline
    Postprocessing & Boolean & no & no & Indicates whether postprocessing should be run. Default: no postprocessing\\
    \hline
    ProcessingScope & String & no & no & The UIMA annotation type  over which to iterate for doing the sentence segmentation. If nothing is given, the document text from the CAS is taken as scope! This is recommended as default!\\
    \hline
  \end{tabular}
  \caption{Parameters of this AE.}
  \label{tab:parameters}
\end{table}


\paragraph{Parameter Settings}
\label{sss:param_settings}
% adapt to your needs

The specific parameter settings are filled in here. For each of the
parameters defined in \ref{sss:parameters}, add the respective values
here (has to be done at least for each parameter that is defined as
mandatory). See Table \ref{tab:param_settings} for the respective
parameter settings of this AE.

\begin{table}[h!]
  \centering
  \begin{tabular}{|p{4cm}|p{4cm}|p{7cm}|}
    \hline
    Parameter Name & Parameter Syntax & Example \\
    \hline\hline
    ModelFilename & full path & \url{resources/JSBD-2.0-biomed.mod.gz}\\
    \hline
    Postprocessing & true/false & true\\
    \hline 
    ProcessingScope & full class name to annotation type & \url{de.julielab.jules.paragraph} (assuming you downloaded the document structure part of the JULIE Lab Type System). If you don't know what to do here, leave it blank!\\
    % add settings for parameters defined above
  \hline
  \end{tabular}
  \caption{Parameter settings of this AE.}
  \label{tab:param_settings}
\end{table}

\paragraph{Type System}
\label{sss:type_system}
On this page, go to \emph{Imported Type} and add the following layers
of the \emph{JULIE UIMA Type System} (Use ``Import by Location''):
\url{julie-basic-types.xml} and
\url{julie-morpho-syntax-types.xml}. If you use the
\textit{ProcessingScope} parameter make sure that the respective
type/type system is also included.


\paragraph{Capabilities}
\label{sss:capabilities}
The sentence splitter only returns annotations from type \url{de.julielab.jules.types.Sentence}. See Table \ref{tab:capabilities}.
% adapt if needed
\begin{table}[h!]
  \centering
  \begin{tabular}{|p{5cm}|p{2cm}|p{2cm}|}
    \hline
    Type & Input & Output \\
    \hline\hline
     de.julielab.jules.types.Sentence & &  $\surd$  \\
      \hline
  \end{tabular}
  \caption{Capabilities of this AE.}
  \label{tab:capabilities}
\end{table} 

\paragraph{Index}
% adapt if needed
Nothing needs to be done here.

\paragraph{Resources}
% adapt if needed
Nothing needs to be done here.




%------------
\clearpage
\section{JSBD: Core Functionality and Stand Alone Tool}
\label{sec_objective}

JULIE Sentence Boundary Detector (JSBD) is a sentence splitter
developed and optimized for the bio-medical domain. In contrast to
most other sentence splitters which consists of simple patterns, JSBD
is based on machine learning (see Section \ref{sec:background}) which
enables it to handle also tricky cases occuring frequently in life
science documents. See \cite{tomanek.medinfo07} for a more in depth
description of JSDB and a performance study.


JSBD offers the following functionalities:
\begin{itemize}
\item training a model
\item prediction using a previously trained model
\item evaluation
\end{itemize}

% \subsection{Changelog}
% since version 1.5:
% \begin{itemize}
% \item models now compressed with GZIP\end{itemize}


% since version 1.2:
% \begin{itemize}
% \item regular expression set modified (JSBD can now also handle german
%   umlauts)\end{itemize}


\subsection{Installation}
Just unpack the tar-ball. The program is written in Java\footnote{Java
  is a registered trademark of Sun Microsystems, Inc.}. Note that JSBD
was only tested with Java 1.5; you need at least the Java 1.5 runtime
environment installed on your system to run JSBD. In addition to the
common Java libraries, JSBD employs \textsc{MALLET} \cite{mallet}, a
machine learning toolkit (no further installation steps are
required here).

\subsection{File Formats}
\label{sec_formats}
There are two input formats for text documents. There are some example
documents in directory \url{testdata}. All data need to be simple
ASCII.
\begin{itemize}
\item For training and evaluation, JSBD needs to know the sentence
  boundaries.  Therefore the documents must have exactly one sentence
  per line. (see directory \url{testdata/train/}).
\item For sentence splitting, the only requirement for the documents
  is that they are plain text, without any XML tags etc. (see
  \url{testdata/split/})
\end{itemize}


\subsection{Using JSBD}

To execute JSBD just type 

\begin{verbatim}
./runJSBDpackaged.sh
\end{verbatim}

without any arguments. This will print the following list of available
modes:

\begin{verbatim}
usage: JSBD <mode> {mode_specific_parameters}
different modes:
c: check texts
t: train a sentence splitting model
p: do the sentence splitting
s: evaluation with 90-10 split
x: evaluation with cross-validation
e: evaluation on previously trained model
\end{verbatim}

When running JSBD only with the mode as its only parameter it will
return the specific parameters needed for this mode.

\subsubsection{Data Check}

The provided data is checked for the correct format. In training mode,
all training files need to be in the following format: one sentence
per line. At the end of the line there should be a EOS
(end-of-sentence) symbol (is defined in the EOSSymbols class). If this
is not the case, an error message is thrown.

\begin{verbatim}
./runJSBDpackaged.sh c

-> usage: JSBD c <textDir>
\end{verbatim}

\textit{<textDir>} is the directory with the text documents to be
checked, all files in this directory are considered


\subsubsection{Evaluation}

There are two evaluation modes to evaluate the sentence splitter on
given training material. Performance is measured in terms of accuracy, i.e.\
the number of correct decisions divided by the total number of
decisions being made.

\begin{description}
\item [90-10 split evaluation] the given data is split into two data
  sets, 90\% of the files are used for training the model, the other
  10\% are used to evaluate the trained model (splits are made on file
  level, i.e. you should make sure, that the files are more or less of
  the same size)

\begin{verbatim}
./runJSBDpackaged.sh s

-> usage: JSBD s <textDir> <errorFile>
\end{verbatim}

\textit{<textDir>} is the directory with the text documents used for
evaluation (i.e. the same format as the training data). All files in
this directory are considered. \textit{<errorFile>} is a file where
all predictions errors are written to.


\item[X-fold cross-validation] the given data is split into X data
  sets. X rounds of evaluation are run, in each round X-1 of these data
  sets are used for training and the remaining one is used for
  evaluation. Finally, the results of each round are averaged. (splits
  are made the same way as in 90-10 mode)

x-validation:
\begin{verbatim}
./runJSBDpackaged.sh x

-> usage: JSBD x <textDir> <cross-val-rounds> <errorFile>
\end{verbatim}

\textit{<textDir>} and \textit{<errorFile>} are the same as in 90-10
split mode. \textit{<cross-val-round>} is the number of rounds for
cross-validation. Typically, this might be set to 10.

\end{description}

\subsubsection{Training}

To train a sentence splitter model, you need to provide some training
material (format: see above). The trained model can then be saved to
disk and used for sentence splitting.

\begin{verbatim}
./runJSBDpackaged.sh t

-> usage: JSBD t <trainDir> <modelFilename>
\end{verbatim}

\textit{<trainDir>} is the directory with training data. All files are
considered and should be in the according format (see above).
\textit{<modelFilename>} is the file where the resulting model is
saved.

\subsubsection{Prediction}

To employ the sentence splitter, you need a trained model. You can get
a sentence splitting (trained on our manually compiled training
material for the bio-medical domain (language: english)) from our
website.


\begin{verbatim}
./runJSBDpackaged.sh p

-> JSBD.sh p <inDir> <outDir> <modelFilename>
\end{verbatim}

\textit{<inDir>} is a directory of text documents which should be
sentence splitted. \textit{<modelFilename>} is the file where a
previously model was saved to. The processed texts with one sentence
per line are written to \textit{<outDir>}

\subsection{Background/Algorithms}
\label{sec:background}

JSBD is based on Conditional Random Fields (CRFs) \cite{lafferty01}, a
sequential learning algorithm. 

\subsection{Evaluation Studies and Available Models}

JSBD was developed and optimized for the bio-medical domain. However,
when training it on respective corpora, it may also be used for other
domains.

We have evaluated JSBD on our data which we compiled for the
bio-medical domain. It consists of data from both the GENIA
\cite{ohta02} and the
\textsc{PennBioIE}\footnote{\url{http://bioie.ldc.upenn.edu/}} corpus
and additional material which we took from MedLine abstracts.

Currently, it comprises about 62000 sentences. An accuracy of 99.8\% is
yielded on this data using 10-fold cross-validation.  You will find
the model trained on this data in the directory \url{resources}.

If you run any evaluations on other data, we would be happy to learn
about your experiences and evaluation results with JSBD.





\section{Copyright and License}
% leave unchanged
This software is Copyright (C) 2008 Jena University Language \& Information
Engineering Lab (Friedrich-Schiller University Jena, Germany), and is
licensed under the terms of the Common Public License, Version 1.0 or (at
your option) any subsequent version.

The license is approved by the Open Source Initiative, and is
available from their website at \url{http://www.opensource.org}.

\bibliographystyle{alpha}
\bibliography{literature}


\end{document}
